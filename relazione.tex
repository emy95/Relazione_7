\documentclass[letterpaper,11pt]{article}
\usepackage{science}
\usepackage[utf8]{inputenc}
\usepackage{mathtools}
\usepackage{subcaption}
\usepackage{tikz}
\usepackage{pgf}
\usepackage{pgfplots}
\usepackage{circuitikz}

\title{\textbf{Transistor bipolare:} interruttore}
\author{Canteri Marco, Biasi Lorenzo, Damiani Emily}
\date{\today}  % Can use \today

\begin{document}
\maketitle

\begin{abstract}
Abbiamo studiato il funzionamento di un transistor, analizzandone la condizione di saturazione. Abbiamo poi utilizzato tale transistor come interruttore veloce per far lampeggiare un LED con frequeza a scelta.
\end{abstract}

\begin{body}
\section{Obiettivi}
\begin{itemize}
\item Studiare il funzionamento di un transistor bipolare come interruttore.
\item Analizzare il circuito interruttore transistor, compresa la condizione di saturazione della
corrente collettore.
\end{itemize}
\section{Procedimento}
<<<<<<< HEAD
È stato costruito il circuito in figura (), dove $R_c = 10001 \pm \Omega$, $R_b$ decade variabile e come sorgente di potenziale $V=15$. Variando la resistenza $R_b$ nel circuito di controllo del transistor si è misurata la corrente $i_c$ con il generatore di $V$, con il multimetro si è misurata la corrente $i_b$ e infine con il tester analogico si è misurata la differenza di potenziale ai capi del collettore e dell'emissore del transistor $V_{ce}$ fino a saturazione.

Abbiamo poi costruito il circuito in figura(), collegando la base del transistor a un generatore d'onda quadra.
=======
\begin{figurehere}
\begin{circuitikz}
\draw (0,0) node[ground, rotate=180]{} to [battery] (0,-1.5)
to [R=$R_c$] (0,-3)
to [leD*] (0,-4.5);
\draw
(0,-5.0) node[npn] (npn) {}
(npn.emitter) node[ground]{};
\draw (0,-1.5)
to [short] (-1.5,-1.5)
to [ospst] (-1.5,-3)
to [R=$R_b$] (-1.5,-5.0)
to [short] (npn.base);
\end{circuitikz}
\caption{Transistor come interruttore}
\end{figurehere}
È stato costruito il circuito in figura (), dove $R_c = 1001.0 \pm \Omega$, $R_b$ decade variabile e come sorgente di potenziale $V=15$. Variando la resistenza $R_b$ nel circuito di controllo del transistor si è misurata la corrente $i_c$ con il generatore di $V$, con il multimetro si è misurata la corrente $i_b$ e infine con il tester analogico si è misurata la differenza di potenziale ai capi del collettore e dell'emissore del transistor $V_{ce}$.
>>>>>>> c2081078f353c7f4bae7f6b281a9db2bdca47bac
\end{body}
\end{document}
